%%%%%%%%%%%%%%%%%%%% author.tex %%%%%%%%%%%%%%%%%%%%%%%%%%%%%%%%%%%
%
%%%%%%%%%%%%%%%% Springer %%%%%%%%%%%%%%%%%%%%%%%%%%%%%%%%%%

\title*{GP As If You Meant It:\\An Exercise for Mindful Practice}
\titlerunning{GP As If You Meant It}

% your contribution title if the original one is too long
\author{William A. Tozier}
\authorrunning{Tozier}

%\institute{William A. Tozier}
% Not affiliated with any institution!

\maketitle

\abstract{In this contribution I present a \emph{kata} called ``GP As If You Meant It'', aimed at advanced users of genetic programming. Inspired by \emph{code katas} that are popular among software developers, it's an exercise designed to help participants hone their skills through mindful practice. Its intent is to surface certain unquestioned habits common in our field: to make the participants painfully aware of the tacit \emph{justification} for certain GP algorithm design decisions they may otherwise take for granted. In the exercise, the human players are charged with trying to ``rescue'' an ineffectual but unstoppable GP system (which is the other ``player''), which has been set up to only use ``random guessing''---but they must do so by \emph{incrementally modifying the search process without interrupting it}. The exercise is a game for two players, plus a Facilitator who acts as a referee. The human ``User'' player examines the state of the GP run in order to make amendments to its rules, using a very limited toolkit. The other ``player'' is the automated GP System itself, which adds to a growing population of solutions by applying the search operators and evaluation functions specified by the User player. The User's goal is to convince the System to produce ``good enough'' answers to a target supervised learning problem chosen by the Facilitator. To further complicate the task, the User must also provide the  Facilitator with convincing justifications, or \emph{warrants}, which explain each move she makes. The Facilitator chooses the initial search problem, provides training data, and most importantly is empowered to \emph{disqualify} any of the User's moves if unconvinced by the accompanying warrants. As a result, the User is forced to work around our field's most insidious habit: that of ``stopping it and starting over again with different parameters''. In the process of working within these constraints, the participants---Facilitator and User---are made mindful of the habits they have already developed, tacitly or explicitly, for coping with ``pathologies'' and ``symptoms'' encountered in their more typical work with GP.}

\begin{keywords}
	mindful practice, design process, coding kata, praxis, Mangle of Practice
\end{keywords}
\index{mindful practice}
\index{design process}
\index{coding kata}
\index{praxis}
\index{Mangle of Practice}

%% [version of \today\ at \currenttime]

\subimport{../../../markdown/}{manuscript}




\bibliographystyle{spbasic}
\bibliography{gp-bibliography,tozier}
